\section{Задача I (оцінювання)}

Почергово розглянемо три основнi задачi, якi можна розв’язати за допомогою прихованих марковських моделей. У першому розділі оглянемо задачу оцінювання: за відомою прихованою марковською моделлю $\lambda=(\mu,A,B)$ й наданою протягом часу $t=\overline{0,T}$ послiдовнiстю спостережень $y=(y_0, \ldots , y_T)_{y_t\in F}$ прагнемо визначити імовiрнiсть $P_\lambda(Y = y)$, де $Y=(Y_0, \ldots , Y_T)$. 

Перетин з повною групою подій, де вектор $X=(X_0, \ldots , X_T)$ пробігає всеможливі значення $x=(x_0, \ldots , x_T)_{x_t\in E}$, дає змогу безпосередньо обчислити шукану ймовірність:
\begin{equation*}
    P_\lambda(Y = y)=P_\lambda(Y = y\, \bigcap\, \Omega)=P_\lambda(Y = y\, \bigcap\, \Bigl[\bigsqcup\limits_{x}(X=x)\Bigr]),
\end{equation*}

а отже
\begin{equation*}
    P_\lambda(Y = y)=\sum\limits_{x}P_\lambda(Y = y, X=x)
\end{equation*} 

З огляду на властивості лінцюга Маркова, вказаний скінченновимірний розподіл виражатиметься через параметри заданої моделі $\lambda$ таким чином:
\begin{equation*}
    P_\lambda(Y = y)=\sum\limits_{x}\mu_{x_0}\prod\limits_{t=0}^{T-1}A_{x_t x_{t+1}}\prod\limits_{t=0}^{T}B_{x_t y_t} 
\end{equation*}

Проте, за допомогою алгоритмів прямого та зворотного ходу шукану ймовірність можна обчислити іншим, більш ефективним способом. 

\subsection{Алгоритм прямого ходу}

Алгоритм прямого ходу полягає у рекурсивному обчисленні сумісних імовірностей часткової послідовності спостережень до моменту часу $t$, перебуваючи в стані $x_t$ в момент часу $t:$
\begin{equation}
    \alpha_t(x_t)=P_\lambda(Y_0=y_0,\ldots,Y_t=y_t,X_t=x_t) \label{formula: alpha}
\end{equation}

Відтак крок за кроком, поклавши $\forall x_0 \in E: \alpha_0(x_0)=\mu_{x_0}B_{x_0 y_0}$, знайдемо значення усіх наступних коефіцієнтів за допомогою такого рекурентного співвідношення:
\begin{equation*}
    \alpha_{t+1}(x_{t+1})=\sum\limits_{x_t\in E}\alpha_t(x_t)\, A_{x_t x_{t+1}}B_{x_{t+1}y_{t+1}} 
\end{equation*}

Тоді загальна ймовірність записуватиметься так:
\begin{equation*}
    P_\lambda(Y = y)=\sum\limits_{x_T \in E}P_\lambda(Y_0=y_0,\ldots,Y_T=y_T,X_T=x_T)=\sum\limits_{x_T \in E}\alpha_T(x_T)
\end{equation*}

\subsection{Алгоритм зворотного ходу}

В алгоритмі зворотного ходу ймовірність $P_\lambda(Y = y)$ шукатиметься через рекурсивне обчислення умовних імовiрностей послiдовностi спостережень вiд моменту часу $t + 1$ до $T$ при умовi, що прихованим є стан $x_t$ в момент часу $t:$
\begin{equation}
    \beta_t(x_t)=P_\lambda(Y_{t+1}=y_{t+1},\ldots,Y_T=y_T \, | \, X_t=x_t) \label{formula: beta}
\end{equation}

Поклавши $\forall x_T \in E: \beta_T(x_T)=1$, віднайдемо значення усіх попередніх коефіцієнтів за допомогою такого рекурентного співвідношення:
\begin{equation*}
    \beta_t(x_t)=\sum\limits_{x_{t+1}\in E}\beta_{t+1}(x_{t+1})\, A_{x_t x_{t+1}}B_{x_{t+1}y_{t+1}} 
\end{equation*}

Відтак загальна імовірність матиме такий вид:
\begin{equation*}
    P_\lambda(Y = y)=\sum\limits_{x_0 \in E}\beta_0(x_0)\, \mu_{x_0}B_{x_0 y_0}
\end{equation*}

\subsection{Приклад: кмітливі учні}

Розглянемо приклад розв'язування задачі оцінювання за допомогою алгоритмів прямого та зворотного ходу:

\vspace{0.2cm}
\begin{mdframed}[style=text box, topline=false, bottomline=false, rightmargin=0cm, leftmargin=0cm]
    \hspace{\tabsize}
    Нехай шкільний вчитель математики протягом п'яти робочих днів задає учням домашнє завдання трьох рівнів складності: легке, складне та підвищеної складності. Водночас тип заданого завдання залежить від піднесеного, нейтрального чи поганого настрою вчителя. Як досвідчений професіонал, вчитель явно не демонструє свій настрій учням. 
    
    Припустимо, що учні бажають відвідати футбольний матч, який відбудеться наступного тижня у середу. Навантаження складним домашнім завданням у ці дні було б вкрай дошкульним. В той же час, повністю розвантажити тиждень від складних завдань не вийде -- викладач, скоріше за все, все ж подбає про баланс різних типів задач протягом тижня.
    
    Зібравши в уважних та кмітливих старшокласників інформацію про ймовірність зміни настрою вчителя, а також спостережуваний рівень складності заданого додому завдання в залежності від настрою викладача, учні починають дослідження: яка ймовірність спостереження наступного тижня послідовності заданих домашніх завдань бажаного рівня складності?
\end{mdframed}

Формалізуємо задачу учнів таким чином: пара $\left\{(X_t,Y_t)\right\}_{t=\overline{0,5}}$ -- прихована марковська модель, де $\left\{ X_t \right\}_{t=\overline{1,5}}$ є прихованою послідовністю настроїв вчителя, заданих на множині станів $E=\left\{\text{\faSmile[regular], \faMeh[regular], \faFrown[regular]}\right\}$, а $\left\{ Y_t \right\}_{t=\overline{1,5}}$ є спостережуваною послідовністю типів завдань, заданих на множині станів $F=\left\{ \text{\faMinus},\ \text{\faRandom},\ \text{\faSitemap} \right\}$. Cкладемо матриці $A$, $B$ та вектор $\mu$ згідно введених позначень:

\vspace{0.4cm}
\begin{table}[H]
    \begin{minipage}[H]{0.35\linewidth}
        \begin{center}
            \begin{tabular}{c|ccc}
                & \faSmile[regular] & \faMeh[regular] & \faFrown[regular] \\
                \hline
                \faSmile[regular] & 0.2 & 0.3 & 0.5 \\
                \faMeh[regular] & 0.2 & 0.2 & 0.6 \\
                \faFrown[regular] & 0 & 0.2 & 0.8 \\
            \end{tabular}
        \end{center} \centering матриця $A$
    \end{minipage}
    \hfill
    \begin{minipage}[H]{0.35\linewidth}
        \begin{center}
            \begin{tabular}{c|ccc}
                & \text{\faMinus} & \text{\faRandom} & \text{\faSitemap} \\
                \hline
                \faSmile[regular] & 0.7 & 0.2 & 0.1 \\
                \faMeh[regular] & 0.3 & 0.4 & 0.3 \\
                \faFrown[regular] & 0 & 0.1 & 0.9 \\
            \end{tabular}
        \end{center} \centering матриця $B$
    \end{minipage}
    \hfill
    \begin{minipage}[H]{0.2\linewidth}
        \begin{center}
            \begin{tabular}{c|c}
                \faSmile[regular] & 0.05 \\
                \hline
                \faMeh[regular] & 0.2 \\
                \hline
                \faFrown[regular] & 0.75 \\
            \end{tabular}
        \end{center} \centering вектор $\mu$
    \end{minipage}
\end{table}

\vspace{0.4cm}
Сформулюємо задачу оцінювання: за заданою моделлю $\lambda=(\mu, A, B)$ визначити ймовірність спостереження такої послідовності:
\begin{equation*}
    P_\lambda(Y_1=\text{\faSitemap},Y_2=\text{\faMinus},Y_3=\text{\faMinus},Y_4=\text{\faRandom},Y_5=\text{\faRandom})
\end{equation*}

Програматично реалізацію алгоритмів прямого та зворотного ходу наведено у Лістингах нижче (спостережені типи завдань перекодовано номерами від $1$ до $3$):

\begin{lstlisting}[firstnumber=1, label = code: alpha, caption = Алгоритм прямого ходу]
    def alpha_calculation(y,m,A,B):
        time = len(y)
        alpha = [[0.0 for i in range(len(B))] for t in range(time)]

        for t in range(time):
            for i in range(len(B)):
                if t == 0: 
                    alpha[t][i] = m[i]*B[i][y[t]-1]
                else:
                    aA = 0
                    for j in range(len(B)):
                        aA += alpha[t-1][j]*A[j][i]
                    alpha[t][i] = aA*B[i][y[t]-1]

        P_alpha = 0
        for i in range(len(alpha[time-1])):
            P_alpha += alpha[time-1][i]

        return alpha, P_alpha
\end{lstlisting}

\vspace{0.4cm}
В результаті роботи алгоритму прямого ходу отримано таке значення шуканої імовірності:
\begin{equation*}
    P_\lambda(Y_1=\text{\faSitemap},Y_2=\text{\faMinus},Y_3=\text{\faMinus},Y_4=\text{\faRandom},Y_5=\text{\faRandom}) = 0.0003985588
\end{equation*}

\begin{lstlisting}[firstnumber=1, label = code: beta, caption = Алгоритм зворотного ходу]
    def beta_calculation(y,m,A,B):
        time = len(y)
        beta = [[0.0 for i in range(len(B))] for t in range(time)]

        for t in range(time-1, -1, -1):
            for i in range(len(B)):
                if t == time-1: 
                    beta[t][i] = 1
                else:
                    sum = 0
                    for j in range(len(B)):
                        sum += beta[t+1][j]*A[i][j]*B[j][y[t+1]-1]
                    beta[t][i] = sum

        P_beta = 0
        for i in range(len(beta[0])):
            P_beta += beta[0][i]*B[i][y[0]-1]*m[i]

        return beta, P_beta
\end{lstlisting}

\vspace{0.4cm}
Імовірність спостереження заданої послідовності за алгоритмом зворотного ходу дорівнює:
\begin{equation*}
    P_\lambda(Y_1=\text{\faSitemap},Y_2=\text{\faMinus},Y_3=\text{\faMinus},Y_4=\text{\faRandom},Y_5=\text{\faRandom}) = 0.0003985588
\end{equation*}

Як бачимо, імовірності за методом прямого та зворотного ходу тотожні. Наведемо проміжні значення коефіцієнтів $\alpha_t(x_t)$ та $\beta_t(x_t):$ 

\vspace{0.4cm}
\begin{table}[H]
    \label{table: forward/backward algorithms}
    \begin{minipage}[H]{0.48\linewidth}
        \begin{center}
            \begin{tabular}{c|c|c|c}
                $\alpha_t(x_t)$ & \faSmile[regular] & \faMeh[regular] & \faFrown[regular] \\
                \hline
                t=1 & 0.005 & 0.06 & 0.675 \\
                \hline
                t=2 & 0.0091 & 0.0446 & 0 \\ 
                \hline
                t=3 & 0.0075 & 0.0035 & 0 \\ 
                \hline
                t=4 & 0.0004 & 0.0012 & 0.0006 \\ 
                \hline
                t=5 & 0.0001 & 0.0002 & 0.0001 \\
            \end{tabular}
        \end{center} \centering результати алгоритму прямого ходу
    \end{minipage}
    \hfill
    \begin{minipage}[H]{0.48\linewidth}
        \begin{center}
            \begin{tabular}{c|c|c|c}
                $\beta_t(x_t)$ & \faSmile[regular] & \faMeh[regular] & \faFrown[regular] \\
                \hline
                t=1 & 0.0018 & 0.0016 & 0.0004 \\
                \hline
                t=2 & 0.0082 & 0.0073 & 0.0019 \\ 
                \hline
                t=3 & 0.038 & 0.0324 & 0.0272 \\ 
                \hline
                t=4 & 0.21 & 0.18 & 0.16 \\ 
                \hline
                t=5 & 1 & 1 & 1 \\
            \end{tabular}
        \end{center} \centering результати алгоритму зворотного ходу
    \end{minipage}
\end{table}

\vspace{0.4cm}
Аналізуючи отримані результати, спершу зауважимо, що сума ймовірностей спостереження усеможливих комбінацій послідовностей заданих протягом тижня типів задач виходить рівною одиниці:
\begin{equation*}
    \sum\limits_{y_1,y_2,y_3,y_4,y_5\in F}P_\lambda(Y_1=y_1 ,Y_2=y_2,Y_3=y_3,Y_4=y_4,Y_5=y_5) = 1
\end{equation*}

Тому авжеж, учнів засмутить мализна отриманої протягом їхнього дослідження імовірності. Проте при спробі підрахувати сумарну ймовірність спостереження легких задач у вівторок та середу при довільних типах задач в інші дні, результат виявиться значно кращим, але все ще не надто обнадійливим:
\begin{equation*}
    \sum\limits_{y_1,y_4,y_5\in F}P_\lambda(Y_1=y_1 ,Y_2=\text{\faMinus},Y_3=\text{\faMinus},Y_4=y_4,Y_5=y_5) = 0.02035
\end{equation*}

\label{P=0.25559}
Наостанок, надзвичайно невтішним для школярів виявиться той факт, що серед імовірностей спостереження усеможливих послідовностей заданих домашніх завдань, найбільшою виявиться імовірність саме такого ланцюжка:
\begin{equation*}
    P_\lambda(Y_1=\text{\faSitemap},Y_2=\text{\faSitemap},Y_3=\text{\faSitemap},Y_4=\text{\faSitemap},Y_5=\text{\faSitemap}) = 0.25559
\end{equation*}